%Introduction Pragraph
%This should be a discussion of all you have learned about the general background of the problem. This includes, but is not limited to, discussion of existing designs and their limitations, results of surveys, limited calculations to investigate feasibility, patent searches, literature searches, references etc. In short, you will report what you have learned about the relevant topics of concern.

During this endeavor of creating a drone from a mere drawing on a notebook to our group bringing forth a functioning and well-thought-out drone design, we have come across a large amount of information that has helped us develop this project nicely. To begin, our entire design is based off of a hat-like drone from a children's movie, and with the help from our lectures and slides, we were able to get the groundwork started with the development of implicating propellers to give us optimal steering and flight predictability. We had to think about the logistics of a concave drone that would work. For the more complicated piece of the project, such as the propeller, we used a YouTube tutorial to get a baseline for what we needed to do to bring the most crucial step of our drone to life. \par

While the design may seem to be the most important part of the process, there is also a need to leave a way to replicate the part and to be able to confirm that this whole idea can even work. During our discussions as a group, we came to a conclusion that we would need to have force fits between the drone body and the propellers and their connectors to ensure that we would get the best use out of them. We found what the best fits were using our tolerance and fits lecture slide to find the dimensions needed and the tolerances to follow to maximize the parts’ capability. Moreover, we used all of these resources to ensure that we could successfully replicate and consistently uphold this drone to a degree in which it would not only work but could work for mass distribution. \par