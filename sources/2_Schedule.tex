%Introduction Pragraph

\subsection{Timeline Status Report}
%Report on the status of a specific phase, deliverable, or task. Include the overall completion percentage for your entire project, as well as each project phase.
    So far in our project, we have completed the modeling for all parts in our project and estimate we are at 60\% completion. Next week, following Exam \#2, we will begin the assembly and drawings of DOR-15 and submit for final approval. Each team member will contribute to the final assembly and drawing creation of our drone.\par

\subsection{Upcoming Tasks and Milestones}
%This might feel redundant based on what you highlighted in the Summary section, but think of it as just another way to list important milestones - or even upcoming holidays or events - that you need everyone to note in your project status update. Take time here to share more detail about the tasks and milestones. The more detail you can provide, the better you will be. Make sure you use the table in the project status report template to call out specific items each time you send an update out to your team. This will help people read and view details easily.


    Some upcoming tasks we have planned are our assembly of our drone on Solidworks as well as drawing creation. We plan to achieve this following the thanksgiving break on 11/28. We want to make sure all our drone parts are done and completed prior to the final deadline to give our team plenty of time to resolve an issues ahead of finals. All main components have been completed and modeled.

\begin{table}[H]
    \begin{center}
    \rotatebox[origin=c]{90}{
    \begin{tabular}{|c|c|l|l|l|l|l|} \hline 
    \textbf{Task Name}& \textbf{Start}& \textbf{End}& \textbf{Duration}& \textbf{Status}& \textbf{Dependent Tasks}& \textbf{Assigned} \\ \hline
        Concept Sketch& 10/30/24& 10/30/24& 00:02:00& Complete& N/A&  Grant/Huy\\ \hline   
        Components Design& 11/07/24& 11/21/24& 14:00:00& Complete& N/A&  All Members\\ \hline
        Assembly& 11/21/24& 11/28/24& 07:00:00& In Progress& Pending&  Grant/Huy/Devin\\ \hline 
        Detailed Drawings&11/07/24& 11/21/24& 14:00:00& In Progress& Pending&  All Members\\ \hline 
        Assembly Drawings&11/21/24& 11/28/24& 07:00:00& In Progress& Pending&  Grant/Huy/Devin\\ \hline 
        Motion Analysis&12/05/24& 12/5/24& 00:05:00& Pending& Pending&  Grant\\ \hline 
        Report&11/07/24& 12/05/24& 14:00:00& Pending& Write Documentation&  All Members\\ \hline 
        Video&12/05/24& 12/05/24& 00:05:00& Pending& N/A&  Devin\\\hline
    \end{tabular}
    }
    \end{center}
    \caption{Project Schedule as of \today}
    \label{tab:table1}
\end{table}

\subsection{Action Items}
%Projects are more than tasks and milestones. You need to track action items to meet those milestones. Use a simple table to track anything and everything that will impact your timeline and budget. Be sure to assign ownership to each action item so everyone understands exactly what's expected of them.

    Our current action items are assembling all parts together and creating drawings for each part. We plan to work on this prior to the presentation on 12/5. Huy, Grant, and Devin will each do the detailed drawings for each part that was made, and then a final assembled drawing will also be made. We expect this to take approx 2 hours but have budgeted for much longer. We then will do the motion analysis and video for the project, explaining DOR-15 and how it would succeed as a potential holiday gift for children. We expect the recording of this video to take approximately 1 hour. Lastly, we will compile a finalized report of our findings and any future improvements that can be made in terms of manufacturing.

\subsection{Project Risks, issues, etc.}
%There's no doubt that things go wrong on projects, but they don’t have to. It’s your job to keep an eye out for issues and risks to make sure things don’t actually go wrong. You’ll want to share as much detail here as possible and be prepared to discuss it.

    Potential issues that we are going to run into is the clearence of the propellars on the hat as well as realistically connecting the motors to a central system that would not interfere with a consumer's head being in the way. We plan to resolve this issue during the assembly and make changes to parts as necessary as we are trying to stay as close to the refernece model as possible. We believe that this drone is capable of being entertaining as well as multifunctional for children and parents.