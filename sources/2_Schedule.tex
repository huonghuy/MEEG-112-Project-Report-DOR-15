%Introduction Pragraph
This week, we completed overlooking and reviewing our hardware as well. We wanted to compensate for a low power and smaller LawnBot. Our research prompted us to look into standardized and modern home-use lawnmowers. For example, the blades and the outer housing to assist in our design. Utilizing multiple sensors (each per a different use) LawnBot will be accurate in reading it's own surroundings. However, the core of our research lies within power consumption, which will be discussed next. \par

\subsection{Material List}
We have drafted a complete material list and are ready for unexpected errors/malfunctions that we may face during the design phase. The project's has taken a turn for the better by decreasing the its size by 1 ft of difference compared to our abstract build. We've removed the concept of using dual blades and applied a singular mower blade for less voltage usage and size decremental, making our project possible without having a crazy powerful battery. We've decided that our project won't have a GPS/RTK Base station, due to the fact that we can simply implement the module to the robot itself, and a 2D LIDAR due to our design choices, leading to less expenses.\par

\begin{table}[H]
    \begin{center}
    \begin{tabular}{|c|c|l|l|l|l|l|} \hline 
    \textbf{Task Name}& \textbf{Start}& \textbf{End}& \textbf{Duration}& \textbf{Status}& \textbf{Dependent Tasks}& \textbf{Assigned} \\ \hline
        Concept Sketch& & & & & &  \\ \hline   
        Components Design& & & & & &  \\ \hline
        Assembly& & & & & &  \\ \hline 
        Detailed Drawings&& & & & &  \\ \hline 
        Assembly Drawings&& & & & &  \\ \hline 
        Motion Analysis&& & & & &  \\ \hline 
        Report&& & & & &  \\ \hline 
        Video&& & & & &  \\\hline
    \end{tabular}
    \end{center}
    \caption{Project Schedule} 
    % We can do either \today if we want to have it constantly updating or we can just make it October 21, 2024
\label{tab:table1}
    
\end{table}

\subsection{Design}
The project's outer design was talked through during our first meeting. Agreeing that its outer shell will take the shape of a cylindrical ring, where the blade and fan are placed at the middle part. This design idea was to excel the vacuum performance of the fan, pulling the grounded grass to then push the cut ones simultaneously. That being said, we sacrificed the LIDAR to embrace its unique design, using a total of 6-8 sonar sensors to try and replicate the LIDAR's dynamic approach. \par

\subsection{Testing Phase}
Realizing the material list's delivery date, we purchased a kit to test on, containing a similar microcontroller, to start the software and wiring testing/implementation. This will speed up the process into understanding the automation field of the project and start the testing phase of our project while helping us plan ahead once we get our actual equipment. \par