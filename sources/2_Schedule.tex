%Introduction Pragraph

\subsection{Timeline Status Report}
%Report on the status of a specific phase, deliverable, or task. Include the overall completion percentage for your entire project, as well as each project phase.
In this project, we will be compiling and utilizing all of our gathered knowledge on part making and use of Solid works to design and assemble a drone which in our case will be DOR-15 or otherwise known as DORIS.  This project is set up so that as a team we can help add to the design and complexity via basing it off of our concept sketch.  Using solid works we will then create a part, detailed drawing and then finally assemble.  To top off this project, we will be writing a report as well as filming a 5 minute video about the process and the part itself.\par

\subsection{Upcoming Tasks and Milestones}
%This might feel redundant based on what you highlighted in the Summary section, but think of it as just another way to list important milestones - or even upcoming holidays or events - that you need everyone to note in your project status update. Take time here to share more detail about the tasks and milestones. The more detail you can provide, the better you will be. Make sure you use the table in the project status report template to call out specific items each time you send an update out to your team. This will help people read and view details easily.
In this project, we will be compiling and utilizing all of our gathered knowledge on part making and use of Solid works to design and assemble a drone which in our case will be DOR-15 or otherwise known as DORIS.  This project is set up so that as a team we can help add to the design and complexity via basing it off of our concept sketch.  Using solid works we will then create a part, detailed drawing and then finally assemble. To top off this project, we will be writing a report as well as filming a 5 minute video about the process and the part itself.

\begin{table}[H]
    \begin{center}
    \rotatebox[origin=c]{90}{
    \begin{tabular}{|c|c|l|l|l|l|l|} \hline 
    \textbf{Task Name}& \textbf{Start}& \textbf{End}& \textbf{Duration}& \textbf{Status}& \textbf{Dependent Tasks}& \textbf{Assigned} \\ \hline
        Concept Sketch& 10/30/24& 10/30/24& 00:02:00& Complete& N/A&  Grant/Huy\\ \hline   
        Components Design& 11/07/24& 11/21/24& 14:00:00& In Progress& Begin&  All Members\\ \hline
        Assembly& 11/21/24& 11/28/24& 07:00:00& Pending& Pending&  Grant/Huy/Devin\\ \hline 
        Detailed Drawings&11/07/24& 11/21/24& 14:00:00& Pending& Pending&  All Members\\ \hline 
        Assembly Drawings&11/21/24& 11/28/24& 07:00:00& Pending& Pending&  Grant/Huy/Devin\\ \hline 
        Motion Analysis&12/05/24& 12/5/24& 00:05:00& Pending& Pending&  Grant\\ \hline 
        Report&11/07/24& 12/05/24& 14:00:00& In Progress& Create Report&  All Members\\ \hline 
        Video&12/05/24& 12/05/24& 00:05:00& Pending& Draft Parts&  Devin\\\hline
    \end{tabular}
    }
    \end{center}
    \caption{Project Schedule as of \today}
    \label{tab:table1}
\end{table}

\subsection{Action Items}
%Projects are more than tasks and milestones. You need to track action items to meet those milestones. Use a simple table to track anything and everything that will impact your timeline and budget. Be sure to assign ownership to each action item so everyone understands exactly what's expected of them.

\subsection{Project Risks, issues, etc.}
%There's no doubt that things go wrong on projects, but they don’t have to. It’s your job to keep an eye out for issues and risks to make sure things don’t actually go wrong. You’ll want to share as much detail here as possible and be prepared to discuss it.